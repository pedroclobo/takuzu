\documentclass[12pt]{article}
\usepackage[a4paper, margin=1in]{geometry}
\usepackage{graphicx}
\usepackage{hyperref}
\usepackage{setspace}
\usepackage{titling}

\setlength{\droptitle}{-2.9cm}

\graphicspath{ {../benchmarks/} }

\title{Inteligência Artificial - Projeto}
\author{Grupo 41}
\date{}

\begin{document}

\maketitle

\setstretch{1.5}

\section{Introdução}
Este projeto da unidade curricular de Inteligência Artificial tem como objetivo
desenvolver um programa em \emph{Python} que resolva o problema
\href{https://en.wikipedia.org/wiki/Takuzu}{Takuzu} utilizando técnicas de
procura de IA.

\section{Resolução de Problemas}
\subsection{Problema de Satisfação de Restrições}
Dada a natureza restritiva do \emph{Takuzu}, decidimos modelar o jogo como um
problema de satisfação de restrições, com o objetivo de reduzir o
\emph{branching factor} da árvore de procura.

Primeiramente, definimos as variáveis como cada uma das posições do tabuleiro.
O domínio de cada uma das variáveis está contido no conjunto $\{0, 1\}$.
As variáveis que representam as posições livres (denominadas
\textbf{variáveis livres}), apresentam esse mesmo domínio, enquanto que as
variáveis que representam uma posição já ocupada têm um domínio unitário,
correspondente ao valor dessa posição.

Durante a procura, são utilizadas as restrições do jogo, listadas no enunciado,
de forma a reduzir os domínios da variáveis livres. Sempre que uma variável
livre vê o tamanho do seu domínio reduzido a 1, a ação correspondente é retornada,
na função \texttt{actions()}. Ao selecionarmos a variável com o menor domínio,
estamos a aplicar a heurística dos \textbf{Valores Remanescentes Mínimos}.

Quando, a partir das restrições, não podemos realizar nenhuma inferência, de
modo a reduzir o domínio das variáveis livres, vê-mo-nos obrigados a \emph{explorar}
as duas possibilidades para os valores daquela posição. Assim, retornamos o
conjunto das duas ações, correspondentes aos valores \texttt{0} e \texttt{1},
fazendo com que a procura \emph{explore} as duas possibilidades.

É importante notar que, a cada nível de profundidade da árvore de procura,
apenas atribuímos valores a uma única variável, visto que a ordem de atribuição
neste problema é irrelevante. Mais explicitamente, atribuir o valor \texttt{1}
à variável \texttt{A} e depois o valor \texttt{0} à variável \texttt{B} é
equivalente a atribuir o valor \texttt{0} à variável \texttt{B} e depois atribuir
o valor \texttt{1} à variável \texttt{A}.

Todos os procedimentos listados acima contribuíram para reduzir o número de nós
gerados, durante a procura.

\subsection{Heurísticas}
A função heurística por nós implementada corresponde ao número de posições
livres no tabuleiro. Esta é uma heurística admissível, pois nunca sobrestima
o custo de chegar ao objetivo. É de notar que o tempo de computação da
heurística é constante, pois é guardado o número de posições livres do tabuleiro
respetivo a cada estado.


\section{Comparação de Procuras}

Comparando as procuras, com recurso aos gráficos seguintes, observamos que a
\texttt{DFS} é o algoritmo que melhor se comporta. Isto seria de esperar porque,
priorizando a procura em profundidade, a procura alcança primeiro o nó solução,
em relação a um procura em largura por exemplo, pois não tem de explorar todos
os nós de profundidades inferiores à profundidade do nó solução.
Embora a \texttt{DFS} não seja nem completa nem ótima, estes fatores não são
relevantes para o problema pois, respetivamente, assume-se que existe
sempre solução e apenas se quer a solução do problema, e não o \emph{caminho}
até à solução.

Uma procura em largura seria teoricamente mais demorada pois, dada a natureza do
problema, sabemos que todas as soluções se encontram à mesma profundidade da
árvore de procura. Logo, expandir a árvore em largura gera todos os nós a
profundidade menor do que a profundidade do nó solução. O facto de a \texttt{BFS}
ser completa e ótima em nada é relevante para o problema, como referido acima.

Conclusões acerca dos algoritmos \texttt{A*} e \texttt{Greedy Search} são difíceis
de inferir pois dependem fortemente da heurística implementada. Observa-se que
as duas procuras têm resultados semelhantes.

De qualquer forma, todas as procuras parecem ser boas opções, sobretudo porque
a função \texttt{actions()} reduz bastante a árvore de procura, tornando-a muito
pequena, o que faz com que os algoritmos se comportem de forma semelhante.

\subsection{Tempo de Execução}
\begin{figure}[h!]
\includegraphics{tempos}
\centering
\end{figure}

\subsection{Nós Gerados}
\begin{figure}[h!]
\includegraphics{gerados}
\centering
\end{figure}

\subsection{Nós Expandidos}
\begin{figure}[h!]
\includegraphics{expandidos}
\centering
\end{figure}

\end{document}
