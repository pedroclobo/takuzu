\documentclass[12pt]{article}
\usepackage[a4paper, margin=1in]{geometry}
\usepackage{hyperref}
\usepackage{setspace}

\title{Inteligência Artificial - Projeto}
\author{Grupo 41}
\date{}

\begin{document}

\maketitle

\setstretch{1.5}

\section{Introdução}
Este projeto da unidade curricular de Inteligência Artificial tem como objetivo
desenvolver um programa em \emph{Python} que resolva o problema
\href{https://en.wikipedia.org/wiki/Takuzu}{Takuzu} utilizando técnicas de
procura de IA.

\section{Resolução de Problemas}
\subsection{Problema de Satisfação de Restrições}
Dada a natureza restritiva do \emph{Takuzu}, decidimos modelar o jogo como um
problema de satisfação de restrições, com o objetivo de reduzir o
\emph{branching factor} da árvore de procura.

Primeiramente, definimos as variáveis como cada uma das posições do tabuleiro.
O domínio de cada uma das variáveis está contido no conjunto $\{0, 1\}$.
As variáveis que representam as posições livres (denominadas
\textbf{variáveis livres}), apresentam esse mesmo domínio, enquanto que as
variáveis que representam uma posição já ocupada têm um domínio unitário,
correspondente ao valor dessa posição.

Durante a procura, são utilizadas as restrições do jogo, listadas no enunciado,
procurando reduzir os domínios da variáveis livres. Sempre que uma variável
livre vê o tamanho do seu domínio reduzido a 1, a ação correspondente é retornada,
na função \texttt{actions()}. Ao selecionarmos a variável com menor domínio,
estamos a aplicar a heurística dos \textbf{Valores Remanescentes Mínimos}.

Quando, a partir das restrições, não podemos realizar nenhuma inferência, de
modo a reduzir o domínio das variáveis livres, vê-mo-nos obrigados a explorar as
duas possibilidades para os valores daquela posição. Assim, retornamos o
conjunto das duas ações correspondentes aos valores 0 e 1, fazendo com que a
procura \emph{explore} as duas possibilidades.

É importante notar que, a cada nível de profundidade da árvore de procura,
apenas atribuímos valores a uma única variável, visto que a ordem de atribuição
neste problema é indiferente. Mais explicitamente, atribuir o valor 1 à variável
A e depois o valor 0 à variável B é equivalente a atribuir o valor 0 à variável
B e depois atribuir o valor 1 à variável A.

Todos os procedimentos listados acima contribuiram imenso para reduzir o número
de nós gerados, durante a procura.

\subsection{Heurísticas}
A função heurística por nós implementada corresponde ao número de posições
livres no tabuleiro. Esta é uma heurística admissível, pois nunca sobrestima
o custo de chegar ao objetivo.


\section{Comparação de Procuras}

\end{document}
